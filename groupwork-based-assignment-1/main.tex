% !TEX TS-program = pdflatex
% !TEX encoding = UTF-8 Unicode

% This is a simple template for a LaTeX document using the "article" class.
% See "book", "report", "letter" for other types of document.

\documentclass[11pt]{article} % use larger type; default would be 10pt
% Copyright 2004 by Till Tantau <tantau@users.sourceforge.net>.
%
% In principle, this file can be redistributed and/or modified under
% the terms of the GNU General Public License, version 2.
%
% However, this file is supposed to be a template to be modified
% for your own needs. For this reason, if you use this file as a
% template and not specifically distribute it as part of a another
% package/program, I grant the extra permission to freely copy and
% modify this file as you see fit and even to delete this copyright
% notice. 


% \mode<presentation>
% {
%     \usetheme{Warsaw}
%     % or ...

%     \setbeamercovered{transparent}
%     % or whatever (possibly just delete it)
% }


\usepackage[english]{babel}
% or whatever

\usepackage[utf8]{inputenc}
% or whatever

\usepackage{times}
\usepackage[T1]{fontenc}
% \usepackage{ctex}
% \usetheme{default}
\usepackage{graphicx}
% 
\usetheme{Madrid}
\usepackage{tikz}
% Or whatever. Note that the encoding and the font should match. If T1
% does not look nice, try deleting the line with the fontenc.

\title{Introduction to Statistics for DS \\ 
Groupwork-Based Assignment 1}
% 
\author{Durham Student ID: \textbf{001102553}, IT account: \textbf{bjsn39}}
\begin{document}
\maketitle

% % Question 1 with 4 multi-choice
\section{Question 1}
\paragraph{\textcolor{red}{I've highlight the keywords and calculation steps in the questions part.}}
\paragraph{1. B}
\paragraph{2. A}
\paragraph{3. B}
\paragraph{4. A}
\paragraph{5. D}

% \section{Question 2}
\subsection{Find the modal tooth length across the entire data set.}
\begin{lstlisting}[style=rstyle]
# clear the console area
cat("\014")

# clear the environment var area
rm(list = ls())

# Load the Tooth Growth data set
data(ToothGrowth)

# Find the modal tooth length
modal_tooth_lengt = as.numeric(names(sort(
    table(ToothGrowth$len), decreasing = TRUE)[1]))
\end{lstlisting}
% \paragraph{Output: modal_tooth_lengt}
\paragraph{Env Output: $modal\_tooth\_length=26.4$}
\subsection{Mean tooth length for guinea pigs who were given their vitamins via orange juice}
% 
\begin{lstlisting}[style=rstyle]
# Calculate the mean tooth length for guinea pigs 
# given vitamins via orange juice
mean_tooth_length_orange = mean(
    ToothGrowth$len[ToothGrowth$supp == "OJ"])
# mean_tooth_length_orange    
\end{lstlisting}
\paragraph{Env Output: $mean\_tooth\_length\_orange=20.66333$}
% 
\subsection{Create a side-by-side box-and-whisker plot:}
% 
\begin{lstlisting}[style=rstyle]
# Create a side-by-side box-and-whisker plot
# for tooth length by supplement type
boxplot(len ~ supp, data = ToothGrowth, 
        col = c("lightblue", "lightgreen"),
        xlab = "Supplement Type", ylab = "Tooth Length",
        main = "Tooth Length by Supplement Type")
\end{lstlisting}
% 
% 
\begin{figure}[H]
    \centering
    \includegraphics[width=0.8\textwidth]{img/ToothLength.pdf}
    % \caption{}
    % \label{fig:your-label}
\end{figure}
% 
% 
\subsection{Comment on which vitamin delivery approach is more effective:}
\paragraph{\textbf{Analytics}}
\begin{itemize}
    \item OJ has a higher \textbf{Median Value} than VC.
    \item OJ data is basically normally distributed, while VC is skewed.
    \item The upper and lower \textbf{quartiles} of VC's span longer distances, so the data is more dispersed.
    \item \textbf{The length of the box:} I don't think there's much difference between the two, and the degree of centralization in the data set is basically the same.
\end{itemize}
\paragraph{\textbf{Conclusion} I think OJ supplement is more effective.}
% \section{Question 3}
\subsection{Expected number of party poppers}
$$ E(X)=Probability\ of\ failure * Total\ number\ of\ party\ poppers $$
$$ E(X)=0.6\% *200=1.2 $$
% 
\subsection{Distribution of fail poppers}
\subsubsection{Poisson distribution}
\paragraph{To represent the number of party poppers that fail to go off in a box as a random variable, X, we can use the Poisson distribution. In this case, the Poisson distribution is appropriate because it models the number of events (party poppers failing to go off) that occur in a fixed interval (a box of party poppers) when the events are rare and random.}
% 
\paragraph{The parameter, $\lambda$ (lambda), is the average number of events in the fixed interval. Here, $\lambda$ is equal to the expected number of party poppers that will fail to go off, which was calculated in Part 1.}
% 
% 
\subsubsection{The assumptions that need to hold for justifying the use of the Poisson distribution}
\begin{itemize}
    \item Events (party poppers failing to go off) occur randomly.
    \item Events are rare, meaning the probability of more than one event occurring in a very short time period is negligible.
    \item Events are independent of each other.
\end{itemize}
% 
% 
% 
\subsection{Distribution of mistake box}
\subsubsection{Distribution for Y}
\paragraph{For the random variable Y, representing the number of party poppers that fail to go off in a box of 2 (due to the administrative error), you can still use the Poisson distribution with the same $\lambda$ value as calculated in Question 3.1, which is based on the company's claim.}

\subsubsection{$E(Y)$ and $Var(Y)$}
\paragraph{According to Poisson Distribution, PMF is:}
$$ P(Y=k)=\frac{e^{- \lambda} \lambda^k}{k!} $$
$$ E(Y)= \mu = \lambda $$
$$ Var(Y) = \sigma^2 = \lambda $$
% \section{Question 4}
\subsection{Calculate $P(X=4)$}
\paragraph{Poisson probability formula is: $ P(X=k)=\frac{e^{- \lambda} \lambda^k}{k!} $}
\paragraph{In this case, $k=4$ and $\lambda=0.4$:}
$$ P(X=4)=\frac{e^{-0.4}*0.4^4}{4!} $$
\subsection{Calculate $P(X<3)$}
$$ P(X<3)=P(X=0)+P(X=1)+P(X=2)$$
\subsection{Calculate $P(Y=4 | Y \ge 3)$}
\paragraph{The conditional probability of an event A given event B is defined as:}
$$ P(A|B)=\frac{P(AB)}{B}$$
\paragraph{The $P(Y=4 | Y \ge 3)$ equals to $\frac{P(Y=4 \cap Y \ge 3)}{P(Y \ge 3)}$}, $P(Y=4 \cap Y \ge 3) = P(Y=4)$.
\paragraph{$P(Y \ge 3)=1-P(Y < 3)$}
\paragraph{Throughout my 4-week journey in the ISDS module, I have embarked on a compelling exploration of groupwork. This assignment serves as an opportunity to reflect on the profound impact of groupwork, recounting experiences that have not only broadened my understanding but also reshaped my perspective on collaborative endeavors with classmates from different countries.}
% 
% 
\section{A Contribution That Benefited Our Group}
% 
\paragraph{Before the start of the second week, I checked the group assignment results on the Ultra platform and was excited to find out my group members. I was the first to reach out to my group members by sending a self-introduction message in the Messages section. This message included my basic personal information and a link to my personal homepage. I was pleased to take the initiative, and my group members were very welcoming. In the days leading up to the start of the class, everyone introduced themselves in the chat box. One of my group members even praised my well-crafted GitHub page. Subsequently, we established a WhatsApp group, which laid a strong foundation for future communication.}
% 
\paragraph{During the second week's ice-breaking activity, our prior interactions proved invaluable. We all participated actively in the activity, and there was no hint of awkwardness, thanks to the positive rapport we had established earlier.}
% 
% 
% 
% 
\section{My Belbin Role in Group Work}
% 
\paragraph{Regarding my role within the team, I believe I embody both the \textbf{Completer Finisher} (CF) and \textbf{Specialist} (SP) roles.}
% 
\paragraph{Firstly, as a Completer Finisher, I consistently deliver high-quality results in any form of assignment. Perhaps owing to my inclination towards perfectionism, I prefer using LaTeX for typesetting (this assignment, for instance, was created using LaTeX), and I utilize specialized software like Tikz and draw.io to produce intricate graphics. As with my first personal assignment, I'm used to using LaTeX to edit formulas instead of taking pictures of my ugly handwritten formulas. Additionally, I play an effective role in motivating the team. For instance, in the process of completing this group-based individual assignment, I proactively scheduled a Zoom meeting with my team members to facilitate discussion and ensure a better outcome.}
% 
\paragraph{Secondly, I also consider myself eligible to be a Specialist. My life's motto is: "If something needs to be calculated more than five times, I will write a program for it rather than performing manual calculations." I intend to uphold the same approach in our group assignments.}
% 
% 
% 
% 
% 
% 
% 
\section{Another Group Member Act in Belbin's role}
% 
\paragraph{It's worth mentioning that there's a notable team member in our group, whom I will refer to as "Z," as it's the initial letter of her name. I believe she embodies the qualities of an excellent Team Worker. In the third week, when a team member from our group decided to switch to another major, she was the first to communicate and confirm this news with the student changing majors. This demonstrated her genuine concern for the well-being of our team members. }
% 
\paragraph{Moreover, whenever a message is posted in our WhatsApp group, she consistently responds with enthusiasm, which is a great way to motivate the team. I consider Z to be a reliable teammate, and I am eager to collaborate with her in the future.}
% 
% 
% 
% 
\section{Lessons from Group Work}
% 
\paragraph{As the deadline for assignment submissions approaches, I often find myself experiencing a sense of anxiety. I tend to meticulously refine our content, even our formatting, in a quest for perfection. Consequently, I have been discussing with our group members the possibility of establishing a group deadline in the future. This would allow ample time for optimization and mitigate potential issues stemming from system errors that could lead to assignment submission failures. The goal is to ensure that when it comes time to submit our work, we do so with confidence.}
% 
% 
% 
% 
% 
% 
% 
% 
% 
% 
% 
% 
% 
% 
\newpage
\begin{mdframed}[
        backgroundcolor=white,  % 背景颜色
        linecolor=black,        % 边框颜色
        leftmargin=5pt,         % 左边距
        rightmargin=5pt,        % 右边距
        linewidth=2pt           % 边框的宽度
    ]
    \textbf{Reflecting on My Group Experience: Insights and Observations}
    \begin{itemize}
        \item I consider myself fortunate to have observed that, thus far, our group members have exhibited remarkable enthusiasm and the willingness to voice their opinions in group work. A recent Zoom discussion we had regarding this assignment is a testament to this positivity. Only one member was absent due to health reasons, but they promptly explained their situation in our WhatsApp group and inquired about the discussion content. I believe we have fostered a conducive group atmosphere. If we can sustain this atmosphere, I am confident we will achieve commendable results in our upcoming group tasks.
        \item As a newcomer to R programming, despite having prior experience with languages like Python and JavaScript, I still grapple with certain programming paradigms. Thankfully, our group member "C" has been exceptionally eager to help me understand and has been quick to address my queries during seminars. I am grateful for their support.
    \end{itemize}
    % 
\end{mdframed}
% 
% 
% 
% 
% 
% 
% 
% 
% 
% 
% 
% 
% 
% 
% 
% 
\begin{mdframed}[
    backgroundcolor=white,  % 背景颜色
    linecolor=black,        % 边框颜色
    leftmargin=5pt,         % 左边距
    rightmargin=5pt,        % 右边距
    linewidth=2pt           % 边框的宽度
]
\textbf{Constructive Criticism for Improvement: }
Initially, we found it challenging to complete the worksheets during the seminars, especially for individuals like us who are newcomers to R programming. Balancing statistical theory knowledge and R language coding proved to be demanding. However, our team wisely heeded the advice of another high-performing group member, who happened to be my friend. Although we were not in the same group, he generously shared his experiences. His recommendation was to first complete the online course on Ultra and try to address the worksheet questions before the classes. Following this advice, we found ourselves much better prepared and at ease by Week 4.
% 
\end{mdframed}
% 
% 
% 
% 
% 
% 
% 
% 
% 
% 
% 
\end{document}
